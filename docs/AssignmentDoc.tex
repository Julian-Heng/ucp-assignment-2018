\documentclass[a4paper, 12pt, titlepage]{article}

\usepackage[
    a4paper,
    lmargin=25.4mm,
    rmargin=25.4mm,
    tmargin=20mm,
    bmargin=20mm
]{geometry}

\usepackage{color}
\usepackage{enumitem}
\usepackage{fancyhdr}
\usepackage{inconsolata}
\usepackage{listings}
\usepackage{listing}
\usepackage{nameref}
\usepackage{parskip}
\usepackage{tocloft}

\newcommand{\code}[1]{\small\texttt{#1}\normalsize}

\definecolor{codegray}{gray}{0.9}
\definecolor{dkgreen}{rgb}{0,0.6,0}
\definecolor{gray}{rgb}{0.5,0.5,0.5}
\definecolor{mauve}{rgb}{0.58,0,0.82}

\lstset
{
    % -- Settings -- %
    breakatwhitespace=true,
    breaklines=true,
    columns=fixed,
    showstringspaces=false,
    xleftmargin=0.65cm,
    % -- Looks -- %
    basicstyle=\footnotesize\ttfamily,
    commentstyle=\color{dkgreen},
    keywordstyle=\color{blue},
    numberstyle=\ttfamily\color{gray},
    stringstyle=\color{mauve}
}

\fancyhf{}
\setlength\columnseprule{0.4pt}
\setlength{\parindent}{0pt}

\title{\huge \textbf{UCP Assignment\\Turtle program}}
\author{Julian Heng (19473701)}
\date{October, 2018}

\begin{document}

\maketitle
\tableofcontents
\newpage

\pagestyle{fancy}

\fancyhf[HL]{\footnotesize{UCP Assignment - Turtle Program}}
\fancyhf[FC]{\thepage}
\fancyhf[FL]{\footnotesize{Julian Heng (19473701)}}

\section{Design Philosophy}
\fancyhf[HR]{\footnotesize{Design Philosophy}}

\subsection{Program Flow}

In the main function of \code{turtle.c}, there are many conditionals 
before \code{processCommands()}. This is to ensure that when calling 
\code{processCommands()}, all the checks that preceed it have returned 
TRUE, meaning that the program can continue with drawing without any 
problems. As such, we can assume that the program has successfully checked 
and validated variables before reaching the drawing command, ensuring 
that no problems can occur as the program could not reach that function.

\subsection{Function Structure}

In addition to following the Curtin provided ``C/C++ Coding Guidelines'', 
the functions in the turtle program share the same function structure.

Each function is expected to have all variable declaration on the top, 
followed by setting default values to the variables. After finishing the 
actions in the functions, any local variables that is allocated memory 
in the heap will be freed in the same function they are declared in.

\subsection{Printing}

There are no \code{printf()} statements, instead opting for using 
\code{fprintf()} statements instead. This way, it makes it easier to 
see which messages are being print to stdout or to stderr. All output 
to stdout will use \code{fprintf(stdout, ...)} while all debuging 
information or error messages will use \code{fprintf(stderr, ...)}.

\subsection{Strings}

Most strings in this program are stored in the heap. The only strings 
that uses the stack are strings used to store temporary information, 
such as the command type from the input file. The choice for using 
malloced strings instead of stack strings is to increase reliability 
instead of convenience. This also serves as covering the edge case 
for strings that can exceed the maximum set string length.

\subsection{Boolean values}

Integers are used to represent boolean values. They are used to exit 
out of the program if an error occurs. In loops, boolean variables are 
used to exit out of the loop if an error occurs. Boolean variables in 
functions are set to their default values on the top of the function, then 
modified if a condition is satisfied.

\subsection{Miscellaneous design}

\begin{itemize}[label={--}]
\begin{samepage}
    \item Do not return expressions, i.e \code{return (i \% 2) == 0;}
    \item Do not compare boolean values with the \code{==} operator
    \item Do not use any functions from standard library unless necessary
    \item Use constants for string literals if possible
    \item Prevent the use of magic numbers
    \item Use brackets on if-else statements, even if the conditional 
          statement is one line
    \item Use wrapper functions to \code{malloc} and \code{free} in 
          \code{tools.c}
\end{samepage}
\end{itemize}

\newpage


\section{Conversion From File to Image}
\fancyhf[HR]{\footnotesize{Conversion From File to Image}}

\subsection{My Implementation}

Here is an excerpt from a file of commands the turtle program will accept:

\begin{lstlisting}
PATTERN .

DRAW 1
ROTATE 180
MOVE 1
ROTATE -90
MOVE 1
ROTATE -90

DRAW 2
ROTATE 180
MOVE 2
ROTATE -90
MOVE 1
ROTATE -90

DRAW 3
ROTATE 180
MOVE 3
ROTATE -90
MOVE 1
ROTATE -90

DRAW 4
\end{lstlisting}

Firstly, the program takes the entire file, line by line, and saves it to 
a linked list within memory. Then it would validate each command in the list. 
If the command is invalid, the program will print an error, which includes 
the problematic line and the line number, and exits with an error code of 4. 
If a command is valid, then the program will continue until it has reached 
the end of the linked list.

After all commands are verified and validated, the turtle program will start 
drawing or interpreting the commands in the list. In this case, the linked 
list is used like a queue, where the file contents are added in last, and 
commands are read from the head. When calculating the new x and y coordinates, 
we have to take into consideration of the limitations with drawing on the 
terminal.

For example, drawing from the coordinates (0, 0) to (1, 0) will only draw 
twi character, even though the coordinates provided changed by one. 
Thus whenever we do a \code{MOVE} or \code{DRAW} command, the resulting 
line will always be one unit length too long. To solve this overhead, 
when a \code{MOVE} or \code{DRAW} command is invoked, we calculate the 
new coordinates but using the length provided minus one. Thus, this gives 
us the coordindates of the new set of coordinates, but only one unit away 
in the direction of the current angle. We will need additional temporary 
coordinates variables in order to log the correct coordinates for the log 
file and also keeping track of the correct coordinates used for the next
\code{MOVE} or \code{DRAW} command.

Using the excerpt above, when the program encounter \code{DRAW 2}, 
with the current angle still at 0.0 degrees and coordinates at (0, 0), 
the coordinates for drawing will be (1, 0). After drawing, the coordinates 
will then be updated to (2, 0) for the next \code{MOVE} or \code{DRAW} 
command. The coordinates are needed to be updated so that they get recorded 
in the log file correctly.

\subsection{Alternate ways}

In my implementation, the program reads through the input file and converts 
it to a linked list. The list is then first vaildated, then processed. An 
alternative way to this approach is instead of reading, then validating, 
then processing, we could do these 3 actions line by line in the file. 
This version of the program could potentially be faster as it does not 
utilise a linked list, thus taking less memory. The drawbacks to this 
approach however is that the picture would be half drawn if an error occurs 
and would theoretically be slower due to calling the validating and 
processing functions on each line in file. In my opinion, I prefer my 
approach as it is a much cleaner way of processing these input files, as 
it the functions do return back to main function instead of the file reading 
function.

In terms of calculating the new coordinates, instead of calculating the 
drawing coordinates and updating after executing the command, we calculate 
new coordinates using the full length, but then provide the drawing 
coordinates to the drawing function. This isn't an improvement over my 
implementation as it simply recalculates the drawing coordinates in a 
different order.

\newpage


\section{Functions Descriptions}
\fancyhf[HR]{\footnotesize{Functions Descriptions}}

\subsection{List of Standard Library Functions Used}

\begin{itemize}[label={--}, noitemsep]
    \item \code{math.h}
        \begin{itemize}[label={--}, noitemsep]
            \item \code{sin}
            \item \code{cos}
        \end{itemize}
    \item \code{stdio.h}
        \begin{itemize}[label={--}, noitemsep]
            \item \code{fclose}
            \item \code{fgetc}
            \item \code{fgets}
            \item \code{fopen}
            \item \code{fprintf}
            \item \code{fputc}
            \item \code{fseek}
            \item \code{perror}
            \item \code{sprintf}
            \item \code{sscanf}
        \end{itemize}
    \item \code{stdlib.h}
        \begin{itemize}[label={--}, noitemsep]
            \item \code{malloc}
            \item \code{free}
        \end{itemize}
    \item \code{string.h}
        \begin{itemize}[label={--}, noitemsep]
            \item \code{memset}
            \item \code{strcmp}
            \item \code{strlen}
            \item \code{strncpy}
        \end{itemize}
\end{itemize}

\pagebreak
\subsection{fileIO.c}
\subsubsection{Libraries}

\begin{itemize}[label={--}, noitemsep]
    \item \code{stdio.h}
    \item \code{stdlib.h}
    \item \code{string.h}
\end{itemize}

\subsubsection{readFileToList}

Read a file into a linked list. Useful for getting the contents to a string 
list for use in a program, instead of using the FILE pointer to process 
the file. Used in \code{turtle.c} to get the input file into a linked list.

\subsubsection{appendToFile}

Append a line to a file. This is used in \code{turtle.c} for adding log 
entries to the log file. It already assums that the FILE pointer to the 
log file is already set.

\subsubsection{getFileStats}

Get a file's length and maximum line lenght. Useful for getting information 
needed to create an array to fit the file or checking if the file is valid. 
Used in \code{turtle.c} to get the right dimensions for allocating memory 
for the string array to hold the file contents.

\subsubsection{printFileError}

Print an error message if a file error occurs. Simply a wrapper function to 
format and print a file error. Used in almost all file input and output 
operations for if an error were to occur.

\pagebreak
\subsection{linkedList.c}
\subsubsection{Libraries}

\begin{itemize}[label={--}, noitemsep]
    \item \code{stdlib.h}
\end{itemize}

\subsubsection{initNode}

Create a linked list node to be inserted into a linked list. Required to 
set values into a linked list node for a linked list.

\subsubsection{initList}

Create a linked list. Required to create a linked list for inserting nodes.

\subsubsection{insertFirst}

Insert a linked list node to the head of the linked list. Useful for a 
stack structure.

\subsubsection{insertLast}

Insert a linked list node to the tail of the linked list. Useful for a 
queue structure.

\subsubsection{removeFirst}

Remove a linked list node from the head of the linked list. Useful for a 
stack structure.

\subsubsection{removeLast}

Remove a linked list node from the tail of the linked list. Useful for a 
queue structure.

\subsubsection{peekFirst}

Get the memory address of the first linked list node in the list. Not part of 
the program, used as a generic function in the linked list.

\subsubsection{peekLast}

Get the memory address of the last linked list node in the list. Not part of 
the program, used as a generic function in the linked list.

\subsubsection{getListLength}

Get the number of linked list node in the linked list. Not part of the 
program, used as a generic function in the linked list.

\subsubsection{getListLengthRecurse}

A recursive function to get the number of nodes in the linked list. Used in 
conjunction to \code{getListLength}.

\subsubsection{clearListStack}

Clear all nodes in the linked list. Required as all the linked list nodes are 
allocated in the heap. This version of the clear list function assumes that 
all the pointers in the linked list nodes are stored in the stack, thus it 
does not attempt to free the node values.

\subsubsection{clearListStackRecurse}

A recursive function to clear all the nodes in the linked list. Used in 
conjunction to \code{clearListStack}.

\subsubsection{clearListMalloc}

Clear all nodes in the linked list. Required as all the linked list nodes are 
allocated in the heap. This version of the clear list function assumes that 
all the pointers in the linked list nodes are stored in the heap, thus it 
frees the node values as well as the linked list node. In cases where a 
linked list contains both values stored in the heap and stack, it is 
recommended to manually free the list by removing the linked list node from 
the list.

\subsubsection{clearListMallocRecurse}

A recursive function to clear all the nodes in the linked list. Used in 
conjunction to \code{clearListMalloc}.

\subsubsection{setNextToNode}

Set a linked list node to the next pointer of another linked list node. This 
function is used internally within \code{linkedList.c} as a static function.

\subsubsection{setPrevToNode}

Set a linked list node to the prev pointer of another linked list node. This 
function is used internally within \code{linkedList.c} as a static function.

\subsubsection{isEmpty}

Checks if the linked list contains no linked list nodes. Useful for checking 
iterating through a linked list.

\pagebreak
\subsection{test.c}
\subsubsection{Libraries}

\begin{itemize}[label={--}, noitemsep]
    \item \code{stdio.h}
    \item \code{stdlib.h}
    \item \code{string.h}
\end{itemize}

\subsubsection{testTools}

Test all the non-printing functions in \code{tools.c}. This testing structure 
goes through each edge case for all functions in \code{tools.c}, ensuring 
that it passes with no logical errors and no memory errors.

\subsubsection{testFileIO}

Test mostly the file reading functions in \code{fileIO.c}. Check for all 
file reading edge cases like empty file, non existant file, trailing 
whitespaces, etc.

\subsubsection{testLinkedList}

Test all the functions in \code{linkedList.c}. It is vital to get the linked 
list correct, thus a rigourous test structure is required to ensure that no 
logical errors or memory errors can occur.

\subsubsection{printResult}

A wrapper function to print \code{passed} or \code{failed} depending on the 
evaulated conditional. Purely eye candy.

\subsubsection{header}

A function that prints a message and a line of equal length. Purely eye candy.

\pagebreak
\subsection{tools.h}
\subsubsection{Libraries}

\begin{itemize}[label={--}, noitemsep]
    \item \code{stdio.h}
    \item \code{stdlib.h}
    \item \code{string.h}
\end{itemize}

\subsubsection{initString}

Allocate memory in the heap for a string and then fill the string with null 
terminators. Required as to prevent manually setting null terminator at the 
end of a string and gives a clean state to a string. Used mostly in 
\code{tools.c} for initialising string arrays or in \code{turtle.c} for 
getting filenames to a string.

\subsubsection{initStringWithContents}

Allocate memory in the heap for a string and then copy the contents of the 
imported string to the new string. Useful for initialising a string with a
set string. Used as an alternative to \code{initString}.

\subsubsection{freePtr}

Wrapper function for \code{free}. Does two tasks; free the pointer and 
set it to null. Simply a wrapper, nothing else.

\subsubsection{stringCompare}

Wrapper function for \code{strcmp} so that it returns a sensible boolean 
value. This is simply syntatic sugar.

\subsubsection{upper}

Convert characters in a string to uppercase. This function is primarily
used when converting command names to uppercase so that the program and 
accept mixed case inputs.

\subsubsection{intBound}

Check if an integer is within the bounds of two other integers. Used in 
\code{turtle.c} for validating \code{FG} and \code{BG} commands, as they 
have a lower and upper limit for terminal colors.

\subsubsection{doubleBound}

Check if a double is within th ebounds of two other doubles. Used in 
\code{turtle.c} for validating \code{DRAW}, \code{MOVE} and \code{ROTATE} 
commands, as they provide a real value.

\subsubsection{doubleCompare}

Compares two doubles. Needed due to how doubles are not precise because of 
floating points. Used to check if a double is within range of two doubles 
inclusive.

\subsubsection{doubleCheck}

Check if one double is larger or equal to another double. Needed for 
checking the range of a double in \code{turtle.c}.

\subsubsection{doubleAbs}

Get the absolute value of a double. Required in \code{doubleCompare} as it 
checks the tolerance of the two double it compares.

\subsubsection{doubleMod}

Get the modulus of a double. Required in turtle to check if the angle is 
a right angle. An alternative to \code{fmod} in \code{math.h}.

\subsubsection{doubleRound}

Round a double to the nearest integer. Needed for the \code{line} function 
as that only takes in integer values and the coordinates are real values.

\subsubsection{degToReg}

Converts degrees to radians. Needed when calculating new coordinates as 
the standard math library's \code{sin} and \code{cos} takes in radians 
instead of degrees.

\subsubsection{removeTrailingNewline}

Remove trailing newline in string. Used in \code{fileIO.c} when recording 
each line in a file to an array. An alternative to \code{fabs} in 
\code{math.h}

\subsubsection{countWords}

Count the number of words in string. Useful for either scanning the contents 
of a string to variables, or checking if string contains the right amount of 
inputs we're expecting. Used for command validation in \code{turtle.c}.

\subsubsection{trim}

Remove leading and trailing whitespaces from string. Used for input 
validation in \code{turtle.c} for accepting commands with leading or trailing 
whitespaces.

\subsubsection{printStringArray}

Print each element in a string array. Useful for printing out the contents 
of a string array. Used in \code{turtle.c} for printing out help and version 
messages.

\pagebreak
\subsection{turtle.h}
\subsubsection{Libraries}

\begin{itemize}[label={--}, noitemsep]
    \item \code{stdio.h}
    \item \code{string.h}
    \item \code{math.h}
\end{itemize}

\subsubsection{main}

The main function which starts the entire program. Set's up the structure 
and the flow of events to draw images in the terminal. First few statements 
are validating input file before processing the inputs.

\subsubsection{checkArgs}

Part one of validating inputs. Checks if program has the right amount of 
command line arguments. We're expecting a minimum of one command line 
argument which is the input filename or displaying help messages.

\subsubsection{processArgs}

Part two of validating inputs. This checks if we're calling the help message 
or the version message or we're giving the program an input file.

\subsubsection{processCommands}

The function which does most of the drawing. After the file is successfully 
stored into a string array, the commands are validated executed.

\subsubsection{calcNewPosition}

Calculates new x and y coordinates using the current angle and length.
Used when \code{MOVE} or \code{DRAW} is invoked.

\subsubsection{doNothing}

A function specifically given to \code{MOVE} since we're not printing any 
characters to the screen when moving. We don't use putChar with a whitespace 
because it would erase the drawing on screen.

\subsubsection{putChar}

A function specifically given to \code{DRAW}. Simply places the given 
character onto stdout.

\subsubsection{validateCommands}

Checks all the commands in the command array and determines if they're valid.
Strips any leading or trailing whitespaces and make the command names case 
insensitive.

\subsubsection{printUsage}

Prints the usage message for basic usage.

\subsubsection{printVersion}

Prints the version/build information of the program.


\newpage


\section{Demonstration}
\fancyhf[HR]{\footnotesize{Demonstration}}

\subsection{Help Message}

\begin{lstlisting}
$ turtle --help
Usage: turtle [FILE]
Draw a graphic from commands in FILE
Example: turtle ./picture.txt

Valid commands:

    +---------+-------+-----------------------+
    | Command | Type  | Range                 |
    +---------+-------+-----------------------+
    | ROTATE  | float | -360 to 360 inclusive |
    | MOVE    | float | Positive              |
    | DRAW    | float | Positive              |
    | FG      | int   | 0 to 15 inclusive     |
    | BG      | int   | 0 to 7 inclusive      |
    | PATTERN | char  | Any character         |
    +---------+-------+-----------------------+

Exit values:

    0 - No errors
    1 - Invalid arguments
    2 - Invalid file
    3 - Error writing to log file
    4 - Invalid command in file

\end{lstlisting}

\subsection{Version Message}

\begin{lstlisting}
$ turtle --version
turtle: A terminal drawing program
Written by Julian Heng (19473701)

Compiler      : gcc (Ubuntu 7.3.0-27ubuntu1~18.04) 7.3.0
Compile by    : 19473701@314-buntu
Compile time  : 2018-10-06T23:30:42+08:00
Last Modified : 2018-10-05T21:01:36+08:00
\end{lstlisting}

\subsection{Valid File}

\begin{lstlisting}
$ cat ../test/turtle/pyramid.txt
pattern .

draw 1
rotate 180
move 1
rotate -90
move 1
rotate -90

draw 2
rotate 180
move 2
rotate -90
move 1
rotate -90

draw 3
rotate 180
move 3
rotate -90
move 1
rotate -90

draw 4
rotate 180
move 4
rotate -90
move 1
rotate -90

draw 5
rotate 180
move 5
rotate -90
move 1
rotate -90

draw 6
rotate 180
move 6
rotate -90
move 1
rotate -90

draw 7
rotate 180
move 7
rotate -90
move 1
rotate -90

draw 8
rotate 180
move 8
rotate -90
move 1
rotate -90

draw 9
rotate 180
move 9
rotate -90
move 1
rotate -90

draw 10
rotate 180
move 10
rotate -90
move 1
rotate -90

draw 9
rotate 180
move 9
rotate -90
move 1
rotate -90

draw 8
rotate 180
move 8
rotate -90
move 1
rotate -90

draw 7
rotate 180
move 7
rotate -90
move 1
rotate -90

draw 6
rotate 180
move 6
rotate -90
move 1
rotate -90

draw 5
rotate 180
move 5
rotate -90
move 1
rotate -90

draw 4
rotate 180
move 4
rotate -90
move 1
rotate -90

draw 3
rotate 180
move 3
rotate -90
move 1
rotate -90

draw 2
rotate 180
move 2
rotate -90
move 1
rotate -90

draw 1
rotate 180
move 1
rotate -90
move 1
rotate -90

$ turtle ../test/turtle/pyramid.txt
.
..
...
....
.....
......
.......
........
.........
..........
.........
........
.......
......
.....
....
...
..
.

$ cat ../test/turtle/octo.txt
move 30

rotate 90
move 10
rotate -90

rotate 45
bg 5
pattern #.
 draw 6
bg 0

rotate 45
pattern .
 draw 6

rotate 45
pattern #
 draw 6

rotate 45
pattern .
 draw 6

rotate 45
pattern #
 draw 6

rotate 45
pattern .
 draw 6

rotate 45
pattern #
 draw 6

rotate 45
pattern .
 draw 6

$ turtle ../test/turtle/octo.txt










                        ......
                       #      #
                      #        #
                     #          #
                    #            #
                   #              .
                   .              .
                   .              .
                   .              .
                   .              .
                   .              .
                   .#             #
                     #           #
                      #        ##
                       #      #
                        #......
\end{lstlisting}

\subsection{Invalid File}

\begin{lstlisting}
$ cat ../test/turtle/small_fail.txt
fgg 5
$ turtle ../test/turtle/small_fail.txt
Invalid command on line 1: fgg 5
$ echo $?
4
$ turtle `a non existant file'
Error opening `a non existant file': No such file or directory
$ echo $?
2
\end{lstlisting}

\pagebreak
\subsection{Invalid Command Line Arguments}

\begin{lstlisting}
$ turtle
Usage: turtle [FILE]
Draw a graphic from commands in FILE
Example: turtle ./picture.txt

Valid commands:

    +---------+-------+-----------------------+
    | Command | Type  | Range                 |
    +---------+-------+-----------------------+
    | ROTATE  | float | -360 to 360 inclusive |
    | MOVE    | float | Positive              |
    | DRAW    | float | Positive              |
    | FG      | int   | 0 to 15 inclusive     |
    | BG      | int   | 0 to 7 inclusive      |
    | PATTERN | char  | Any character         |
    +---------+-------+-----------------------+

Exit values:

    0 - No errors
    1 - Invalid arguments
    2 - Invalid file
    3 - Error writing to log file
    4 - Invalid command in file

$ echo $?
1
\end{lstlisting}

\newpage

\end{document}
