\documentclass[a4paper, 12pt, titlepage]{article}

\usepackage[
    a4paper,
    lmargin=25.4mm,
    rmargin=25.4mm,
    tmargin=20mm,
    bmargin=20mm
]{geometry}

\usepackage[english]{babel}
\usepackage{blindtext}

\usepackage{parskip}
\usepackage{listings}
\usepackage{enumitem}
\usepackage{listing}
\usepackage{color}
\usepackage{inconsolata}

\newcommand{\code}[1]{\small\texttt{#1}\normalsize}

\definecolor{dkgreen}{rgb}{0,0.6,0}
\definecolor{gray}{rgb}{0.5,0.5,0.5}
\definecolor{codegray}{gray}{0.9}
\definecolor{mauve}{rgb}{0.58,0,0.82}

\lstset
{
    % -- Settings -- %
    columns=fixed,
    xleftmargin=0.65cm,
    showstringspaces=false,
    breaklines=true,
    breakatwhitespace=true,
    % -- Looks -- %
    basicstyle=\footnotesize\ttfamily,
    numberstyle=\ttfamily\color{gray},
    keywordstyle=\color{blue},
    commentstyle=\color{dkgreen},
    stringstyle=\color{mauve}
}

\title{\huge Turtle program - UCP Assignment}
\author{Julian Heng (19473701)}
\date{October, 2018}

\begin{document}

\maketitle
\tableofcontents
\newpage

\section{Design Philosophy}

\subsection{Program Flow}

In the main function of \code{turtle.c}, there are many conditionals
before \code{processCommands()}. This is to ensure that when calling
\code{processCommands()}, all the checks that preceed 
\code{processCommands()} have returned TRUE, meaning that the program 
can continue with drawing without any problems.

\subsection{Function Structure}

In addition to following the Curtin provided ``C/C++ Coding Guidelines'', 
the functions in the turtle program share the same function structure.

Each function is expected to have all variable declaration on the top, 
followed by setting default values to the variables. After finishing the 
actions in the functions, any local variables that is allocated memory 
in the heap will be freed.

\subsection{Printing}

There are no \code{printf()} statements, instead opting for using 
\code{fprintf()} statements instead. This way, it makes it easier to 
see which messages are being print to stdout or to stderr. All output 
to stdout will use \code{fprintf(stdout, ...)} while all debuging 
information or error messages will use \code{fprintf(stderr, ...)}.

\subsection{Strings}

Most strings in this program are stored in the heap. The only strings 
that uses the stack are strings used to store temporary information, 
such as the command type from the input file. The choice for using 
malloced strings instead of stack strings is to favour reliability 
instead of convenience. This also serves as covering the edge case 
for strings that can exceed the maximum set string length.

\subsection{Boolean values}

Integers are used to represent boolean values. They are used to exit 
out of the program if an error occurs. In loops, boolean variables are 
used to exit out of the loop if an error occurs.

\subsection{Miscellaneous design}

\begin{itemize}[label={--}]
\begin{samepage}
    \item Do not return expressions, i.e \code{return (i \% 2) == 0;}
    \item Do not use ternary operations, i.e \code{a = i == 1 ? 0 : 1;}
    \item Use brackets on if-else statements, even if the conditional 
          statement is one line.
    \item Use provided library functions if possible, i.e \code{isupper()} 
          from \code{ctype.h}.
    \item Don't compare boolean values with the \code{==} operator.
\end{samepage}
\end{itemize}

\newpage


\section{Conversion from file to image}

\subsection{My Implementation}

Here is an excerpt from a file of commands the turtle program will accept:

\begin{lstlisting}
PATTERN .

DRAW 1
ROTATE 180
MOVE 1
ROTATE -90
MOVE 1
ROTATE -90

DRAW 2
ROTATE 180
MOVE 2
ROTATE -90
MOVE 1
ROTATE -90

DRAW 3
ROTATE 180
MOVE 3
ROTATE -90
MOVE 1
ROTATE -90

DRAW 4
\end{lstlisting}

Firstly, the program takes the entire file, line by line, and saves it to
an array within memory. Then it would validate each command in the array. 
If the command is invalid, the program will print an error, which includes 
the problematic line and the line number, and exits with an error code of 
4. If a command is valid, the command is then added to a linked list. The 
linked list in question is used like a queue.

After all commands are validated and added to a linked list, the turtle 
program will start drawing or interpreting the commands in the list. When 
calculating the new x and y coordinates, we have to take into consideration 
of the limitations with drawing on the terminal.

For example, drawing from the coordinates (0, 0) to (0, 0) will only draw 
one character, even though the coordinates provided did not change. 
Thus whenever we do a \code{MOVE} or \code{DRAW} command, the resulting 
line will always be one unit length too long. To solve this overhead, 
when a \code{MOVE} or \code{DRAW} command is invoked, we calculate the 
new coordinates but using the length provided minus one. Thus, this gives 
us the coordindates of the new set of coordinates, but only one unit away 
in the direction of the current angle.

Using the excerpt above, when the program encounter \code{DRAW 2}, 
with the current angle still at 0.0 degrees and coordinates at (0, 0), 
the coordinates for drawing will be (1, 0). After drawing, the coordinates 
will then be updated to (2, 0) for the next \code{MOVE} or \code{DRAW} 
command.

\subsection{Alternate ways}

In my implementation, the program reads through the command list once to 
read it to an array, read it again to validate the commands and then 
reading it once more to execute the command. An alternate way to this 
approach is to instead perform the validation and the execution while reading 
the file. This version of the program would consume less memory as it does 
require an array for storing the file contents, but it would cause the 
program to draw before doing any commands validation.

In terms of calculating the new coordinates, instead of calculating the 
drawing coordinates and updating after executing the command, we calculate 
new coordinates using the full length, but then provide the drawing 
coordinates to the drawing function. This isn't an improvement over my 
implementation as it simply recalculates the drawing coordinates in a 
different order.

\newpage


\section{Functions Descriptions}

\subsection{List of Standard Library Function Used}

\begin{itemize}[label={--}, noitemsep]
    \item \code{ctype.h}
        \begin{itemize}[label={--}, noitemsep]
            \item \code{toupper}
            \item \code{isspace}
        \end{itemize}
    \item \code{math.h}
        \begin{itemize}[label={--}, noitemsep]
            \item \code{fabs}
            \item \code{fmod}
            \item \code{sin}
            \item \code{cos}
        \end{itemize}
    \item \code{stdio.h}
        \begin{itemize}[label={--}, noitemsep]
            \item \code{fprintf}
            \item \code{fopen}
            \item \code{fclose}
            \item \code{fgets}
            \item \code{perror}
            \item \code{fgetc}
            \item \code{fseek}
            \item \code{sscanf}
            \item \code{sprintf}
            \item \code{fputc}
        \end{itemize}
    \item \code{stdlib.h}
        \begin{itemize}[label={--}, noitemsep]
            \item \code{free}
            \item \code{malloc}
        \end{itemize}
    \item \code{string.h}
        \begin{itemize}[label={--}, noitemsep]
            \item \code{strncpy}
            \item \code{strcmp}
            \item \code{memset}
            \item \code{strlen}
        \end{itemize}
\end{itemize}

\subsection{fileIO.c}
\subsubsection{readFileToArray}

Read a file into an array. Useful for getting the contents to a string 
array for use in a program, instead of using the FILE pointer to process 
the file. Used in \code{turtle.c} to get the input file into an array.

\subsubsection{appendToFile}

Append a line to a file. This is used in \code{turtle.c} for adding log 
entries to the log file. It already assums that the FILE pointer to the 
log file is already set.

\subsubsection{getFileStats}

Get a file's length and maximum line lenght. Useful for getting information 
needed to create an array to fit the file or checking if the file is valid. 
Used in \code{turtle.c} to get the right dimensions for allocating memory 
for the string array to hold the file contents.

\subsubsection{printFileError}

Print an error message if a file error occurs. Simply a wrapper function to 
format and print a file error. Used in almost all file input and output 
operations for if an error were to occur.

\subsection{linkedList.c}
\subsubsection{initNode}

\subsubsection{initList}

\subsubsection{insertFirst}

\subsubsection{insertLast}

\subsubsection{removeFirst}

\subsubsection{removeLast}

\subsubsection{peekFirst}

\subsubsection{peekLast}

\subsubsection{getListLength}

\subsubsection{clearListStack}

\subsubsection{clearListMalloc}

\subsubsection{isEmpty}

\subsection{test.c}
\subsubsection{testTools}

Test all the non-printing functions in \code{tools.c}. This testing structure 
goes through each edge case for all functions in \code{tools.c}, ensuring 
that it passes with no logical errors and no memory errors.

\subsubsection{testFileIO}

Test mostly the file reading functions in \code{fileIO.c}. Check for all 
file reading edge cases like empty file, non existant file, trailing 
whitespaces, etc.

\subsubsection{testLinkedList}

Test all the functions in \code{linkedList.c}. It is vital to get the linked 
list correct, thus a rigourous test structure is required to ensure that no 
logical errors or memory errors can occur.

\subsubsection{printResult}

A wrapper function to print \code{passed} or \code{failed} depending on the 
evaulated conditional. Purely eye candy.

\subsubsection{header}

A function that prints a message and a line of equal length. Purely eye candy.

\subsection{tools.h}
\subsubsection{initString}

Allocate memory in the heap for a string and then fill the string with null 
terminators. Required as to prevent manually setting null terminator at the 
end of a string and gives a clean state to a string. Used mostly in 
\code{tools.c} for initialising string arrays or in \code{turtle.c} for 
getting filenames to a string.

\subsubsection{initStringWithContents}

Allocate memory in the heap for a string and then copy the contents of the 
imported string to the new string. Useful for initialising a string with a
set string. Used as an alternative to \code{initString}.

\subsubsection{initStringArray}

Allocate memory in the heap for a string array. Essentially the same as 
\code{initString} but done in a loop for setting up a string array, thus 
creating a string array full of clean strings.

\subsubsection{freePtr}

Wrapper function for \code{free}. Does two tasks; free the pointer and 
set it to null. Simply a wrapper, nothing else.

\subsubsection{freeArray}

Wrapper function for \code{free} on arrays. Same as freePtr but is applied 
to all elements in array. Simply a wrapper, nothing else.

\subsubsection{stringCompare}

Wrapper function for \code{strcmp} so that it returns a sensible boolean 
value. This is simply syntatic sugar.

\subsubsection{upperRange}

Convert characters up to a range to uppercase. This function is primarily 
used when converting command names to uppercase. A range is needed because 
applying uppercase to the entire string can cause the pattern to also change 
to uppercase. This is undesirable, thus using a range prevents this problem.

\subsubsection{doubleCompare}

Compares two doubles. Needed due to how doubles are not precise because of 
floating points. Used to check if a double is within range of two doubles 
inclusive.

\subsubsection{removeTrailingNewline}

Remove trailing newline in string. Used in \code{fileIO.c} when recording 
each line in a file to an array.

\subsubsection{countWords}

Count the number of words in string. Useful for either scanning the contents 
of a string to variables, or checking if string contains the right amount of 
inputs we're expecting. Used for command validation in \code{turtle.c}.

\subsubsection{trim}

Remove leading and trailing whitespaces from string. Used for input 
validation in \code{turtle.c} for accepting commands with leading or trailing 
whitespaces.

\subsubsection{printStringArray}

Print each element in a string array. Useful for printing out the contents 
of a string array. Used in \code{turtle.c} for printing out help and version 
messages.

\subsection{turtle.h}
\subsubsection{main}

\subsubsection{checkArgs}

\subsubsection{processArgs}

\subsubsection{processCommands}

\subsubsection{calcNewPosition}

\subsubsection{doNothing}

\subsubsection{putChar}

\subsubsection{validateCommands}

\subsubsection{printUsage}

\subsubsection{printVersion}


\newpage


\section{Demonstration}

\subsection{Help Message}

\begin{lstlisting}
$ turtle --help
Usage: turtle [FILE]
Draw a graphic from commands in FILE
Example: turtle ./picture.txt

Valid commands:

    +---------+-------+-----------------------+
    | Command | Type  | Range                 |
    +---------+-------+-----------------------+
    | ROTATE  | float | -360 to 360 inclusive |
    | MOVE    | float | Positive              |
    | DRAW    | float | Positive              |
    | FG      | int   | 0 to 15 inclusive     |
    | BG      | int   | 0 to 7 inclusive      |
    | PATTERN | char  | Any character         |
    +---------+-------+-----------------------+

Exit values:

    0 - No errors
    1 - Invalid arguments
    2 - Invalid file
    3 - Error writing to log file
    4 - Invalid command in file

\end{lstlisting}

\subsection{Version Message}

\begin{lstlisting}
$ turtle --version
turtle: A terminal drawing program
Written by Julian Heng (19473701)

Compiler      : gcc (Ubuntu 7.3.0-27ubuntu1~18.04) 7.3.0
Compile by    : 19473701@314-buntu
Compile time  : 2018-10-06T23:30:42+08:00
Last Modified : 2018-10-05T21:01:36+08:00
\end{lstlisting}

\subsection{Valid File}

\begin{lstlisting}
$ cat ../test/turtle/pyramid.txt 
pattern .

draw 1
rotate 180
move 1
rotate -90
move 1
rotate -90

draw 2
rotate 180
move 2
rotate -90
move 1
rotate -90

draw 3
rotate 180
move 3
rotate -90
move 1
rotate -90

draw 4
rotate 180
move 4
rotate -90
move 1
rotate -90

draw 5
rotate 180
move 5
rotate -90
move 1
rotate -90

draw 6
rotate 180
move 6
rotate -90
move 1
rotate -90

draw 7
rotate 180
move 7
rotate -90
move 1
rotate -90

draw 8
rotate 180
move 8
rotate -90
move 1
rotate -90

draw 9
rotate 180
move 9
rotate -90
move 1
rotate -90

draw 10
rotate 180
move 10
rotate -90
move 1
rotate -90

draw 9
rotate 180
move 9
rotate -90
move 1
rotate -90

draw 8
rotate 180
move 8
rotate -90
move 1
rotate -90

draw 7
rotate 180
move 7
rotate -90
move 1
rotate -90

draw 6
rotate 180
move 6
rotate -90
move 1
rotate -90

draw 5
rotate 180
move 5
rotate -90
move 1
rotate -90

draw 4
rotate 180
move 4
rotate -90
move 1
rotate -90

draw 3
rotate 180
move 3
rotate -90
move 1
rotate -90

draw 2
rotate 180
move 2
rotate -90
move 1
rotate -90

draw 1
rotate 180
move 1
rotate -90
move 1
rotate -90

$ turtle ../test/turtle/pyramid.txt
.
..
...
....
.....
......
.......
........
.........
..........
.........
........
.......
......
.....
....
...
..
.
\end{lstlisting}

\subsection{Invalid File}

\begin{lstlisting}
$ cat ../test/turtle/small_fail.txt
fgg 5
$ turtle ../test/turtle/small_fail.txt
Invalid command on line 1: fgg 5
$ echo $?
4
$ turtle `a non existant file'
Error opening `a non existant file': No such file or directory
$ echo $?
2
\end{lstlisting}

\subsection{Invalid Command Line Arguments}

\begin{lstlisting}
$ turtle
Usage: turtle [FILE]
Draw a graphic from commands in FILE
Example: turtle ./picture.txt

Valid commands:

    +---------+-------+-----------------------+
    | Command | Type  | Range                 |
    +---------+-------+-----------------------+
    | ROTATE  | float | -360 to 360 inclusive |
    | MOVE    | float | Positive              |
    | DRAW    | float | Positive              |
    | FG      | int   | 0 to 15 inclusive     |
    | BG      | int   | 0 to 7 inclusive      |
    | PATTERN | char  | Any character         |
    +---------+-------+-----------------------+

Exit values:

    0 - No errors
    1 - Invalid arguments
    2 - Invalid file
    3 - Error writing to log file
    4 - Invalid command in file

$ echo $?
1
\end{lstlisting}

\newpage

\end{document}
